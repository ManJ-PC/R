% Options for packages loaded elsewhere
\PassOptionsToPackage{unicode}{hyperref}
\PassOptionsToPackage{hyphens}{url}
%
\documentclass[
]{article}
\usepackage{amsmath,amssymb}
\usepackage{iftex}
\ifPDFTeX
  \usepackage[T1]{fontenc}
  \usepackage[utf8]{inputenc}
  \usepackage{textcomp} % provide euro and other symbols
\else % if luatex or xetex
  \usepackage{unicode-math} % this also loads fontspec
  \defaultfontfeatures{Scale=MatchLowercase}
  \defaultfontfeatures[\rmfamily]{Ligatures=TeX,Scale=1}
\fi
\usepackage{lmodern}
\ifPDFTeX\else
  % xetex/luatex font selection
\fi
% Use upquote if available, for straight quotes in verbatim environments
\IfFileExists{upquote.sty}{\usepackage{upquote}}{}
\IfFileExists{microtype.sty}{% use microtype if available
  \usepackage[]{microtype}
  \UseMicrotypeSet[protrusion]{basicmath} % disable protrusion for tt fonts
}{}
\makeatletter
\@ifundefined{KOMAClassName}{% if non-KOMA class
  \IfFileExists{parskip.sty}{%
    \usepackage{parskip}
  }{% else
    \setlength{\parindent}{0pt}
    \setlength{\parskip}{6pt plus 2pt minus 1pt}}
}{% if KOMA class
  \KOMAoptions{parskip=half}}
\makeatother
\usepackage{xcolor}
\usepackage[margin=1in]{geometry}
\usepackage{color}
\usepackage{fancyvrb}
\newcommand{\VerbBar}{|}
\newcommand{\VERB}{\Verb[commandchars=\\\{\}]}
\DefineVerbatimEnvironment{Highlighting}{Verbatim}{commandchars=\\\{\}}
% Add ',fontsize=\small' for more characters per line
\usepackage{framed}
\definecolor{shadecolor}{RGB}{248,248,248}
\newenvironment{Shaded}{\begin{snugshade}}{\end{snugshade}}
\newcommand{\AlertTok}[1]{\textcolor[rgb]{0.94,0.16,0.16}{#1}}
\newcommand{\AnnotationTok}[1]{\textcolor[rgb]{0.56,0.35,0.01}{\textbf{\textit{#1}}}}
\newcommand{\AttributeTok}[1]{\textcolor[rgb]{0.13,0.29,0.53}{#1}}
\newcommand{\BaseNTok}[1]{\textcolor[rgb]{0.00,0.00,0.81}{#1}}
\newcommand{\BuiltInTok}[1]{#1}
\newcommand{\CharTok}[1]{\textcolor[rgb]{0.31,0.60,0.02}{#1}}
\newcommand{\CommentTok}[1]{\textcolor[rgb]{0.56,0.35,0.01}{\textit{#1}}}
\newcommand{\CommentVarTok}[1]{\textcolor[rgb]{0.56,0.35,0.01}{\textbf{\textit{#1}}}}
\newcommand{\ConstantTok}[1]{\textcolor[rgb]{0.56,0.35,0.01}{#1}}
\newcommand{\ControlFlowTok}[1]{\textcolor[rgb]{0.13,0.29,0.53}{\textbf{#1}}}
\newcommand{\DataTypeTok}[1]{\textcolor[rgb]{0.13,0.29,0.53}{#1}}
\newcommand{\DecValTok}[1]{\textcolor[rgb]{0.00,0.00,0.81}{#1}}
\newcommand{\DocumentationTok}[1]{\textcolor[rgb]{0.56,0.35,0.01}{\textbf{\textit{#1}}}}
\newcommand{\ErrorTok}[1]{\textcolor[rgb]{0.64,0.00,0.00}{\textbf{#1}}}
\newcommand{\ExtensionTok}[1]{#1}
\newcommand{\FloatTok}[1]{\textcolor[rgb]{0.00,0.00,0.81}{#1}}
\newcommand{\FunctionTok}[1]{\textcolor[rgb]{0.13,0.29,0.53}{\textbf{#1}}}
\newcommand{\ImportTok}[1]{#1}
\newcommand{\InformationTok}[1]{\textcolor[rgb]{0.56,0.35,0.01}{\textbf{\textit{#1}}}}
\newcommand{\KeywordTok}[1]{\textcolor[rgb]{0.13,0.29,0.53}{\textbf{#1}}}
\newcommand{\NormalTok}[1]{#1}
\newcommand{\OperatorTok}[1]{\textcolor[rgb]{0.81,0.36,0.00}{\textbf{#1}}}
\newcommand{\OtherTok}[1]{\textcolor[rgb]{0.56,0.35,0.01}{#1}}
\newcommand{\PreprocessorTok}[1]{\textcolor[rgb]{0.56,0.35,0.01}{\textit{#1}}}
\newcommand{\RegionMarkerTok}[1]{#1}
\newcommand{\SpecialCharTok}[1]{\textcolor[rgb]{0.81,0.36,0.00}{\textbf{#1}}}
\newcommand{\SpecialStringTok}[1]{\textcolor[rgb]{0.31,0.60,0.02}{#1}}
\newcommand{\StringTok}[1]{\textcolor[rgb]{0.31,0.60,0.02}{#1}}
\newcommand{\VariableTok}[1]{\textcolor[rgb]{0.00,0.00,0.00}{#1}}
\newcommand{\VerbatimStringTok}[1]{\textcolor[rgb]{0.31,0.60,0.02}{#1}}
\newcommand{\WarningTok}[1]{\textcolor[rgb]{0.56,0.35,0.01}{\textbf{\textit{#1}}}}
\usepackage{graphicx}
\makeatletter
\def\maxwidth{\ifdim\Gin@nat@width>\linewidth\linewidth\else\Gin@nat@width\fi}
\def\maxheight{\ifdim\Gin@nat@height>\textheight\textheight\else\Gin@nat@height\fi}
\makeatother
% Scale images if necessary, so that they will not overflow the page
% margins by default, and it is still possible to overwrite the defaults
% using explicit options in \includegraphics[width, height, ...]{}
\setkeys{Gin}{width=\maxwidth,height=\maxheight,keepaspectratio}
% Set default figure placement to htbp
\makeatletter
\def\fps@figure{htbp}
\makeatother
\setlength{\emergencystretch}{3em} % prevent overfull lines
\providecommand{\tightlist}{%
  \setlength{\itemsep}{0pt}\setlength{\parskip}{0pt}}
\setcounter{secnumdepth}{-\maxdimen} % remove section numbering
\ifLuaTeX
  \usepackage{selnolig}  % disable illegal ligatures
\fi
\IfFileExists{bookmark.sty}{\usepackage{bookmark}}{\usepackage{hyperref}}
\IfFileExists{xurl.sty}{\usepackage{xurl}}{} % add URL line breaks if available
\urlstyle{same}
\hypersetup{
  pdfauthor={V. Inácio de Carvalho \& M. de Carvalho},
  hidelinks,
  pdfcreator={LaTeX via pandoc}}

\title{Incomplete Data Analysis

Missing Mechanisms---Examples}
\usepackage{etoolbox}
\makeatletter
\providecommand{\subtitle}[1]{% add subtitle to \maketitle
  \apptocmd{\@title}{\par {\large #1 \par}}{}{}
}
\makeatother
\subtitle{School of Mathematics, University of Edinburgh}
\author{V. Inácio de Carvalho \& M. de Carvalho}
\date{}

\begin{document}
\maketitle

Dedicado à ideia de simulação (laboratório de estatística e DS) em forma
de \ldots{} consequências e limitações d eum procedimento e o seu
potencial. Mecanismos de (omissão de dados)\ldots{} e simular dados Here
we will simulate data under different missingness mechanisms. Because we
are working with simulated data we can actually do comparisons between
the complete, observed, and missing data distributions (something that
when working with real data we, obviously, cannot).

Contextos de simulação e contexto de MAR, missing at Random!
Distribuição em forma de símbolo.. \# Blood pressure simulation example

This first example is adapted from Schafer and Graham (2002,
\emph{Psychological Methods}). Suppose that the systolic blood pressure
(SBP) of \(n\) individuals is recorded both in January (coded in
variable \(Y_1\)) and in February (coded in variable \(Y_2\)). We will
impose missing on \(Y_2\) according to the three missingness mechanisms
we have learned, that is, MCAR, MAR, and MNAR.

First of all, we need to generate the simulated data. We will simulate
data from a bivariate normal distribution, with means
\(\mu_1=\mu_2=120\), standard deviations \(\sigma_1=\sigma_2=20\), and
correlation \(\rho=0.6\). This should generate reasonable SBP values. In
\texttt{R}, we need to load the package \texttt{MASS}, which has the
function \texttt{mvrnorm} that will allow us to simulate random numbers
from a bivariate (in general from a multivariate) normal distribution.

\begin{Shaded}
\begin{Highlighting}[]
\FunctionTok{require}\NormalTok{(MASS) }\CommentTok{\# lê um pacote, permite ver visualização/distribuição em forma de sino}
\end{Highlighting}
\end{Shaded}

We will now simulate the complete data for \(Y_1\) (measurement of
systolic blood pressure in January) and \(Y_2\) (measurement of systolic
blood pressure in February). The function \texttt{mvrnorm} needs as
input the number of random pairs we want to generate, the vector of
means (\((\mu_{Y_1},\mu_{Y_2})^{T}\)), and the covariance matrix. For
more information type \texttt{help(mvrnorm)} In the example, we are
given the correlation between \(Y_1\) and \(Y_2\), denoted by
\(\rho_{Y_{1},Y_{2}}\), and their respective standard deviations
(\(\sigma_{Y_{1}}\) and \(\sigma_{Y_{2}}\)), and from this information,
we can easily compute the covariance between \(Y_1\) and \(Y_2\),
denoted by \(\sigma_{Y_{1},Y_{2}}\), by noting that \[
\rho_{Y_{1},Y_{2}}=\frac{\sigma_{Y_{1},Y_{2}}}{\sigma_{Y_{1}}\sigma_{Y_{2}}},
\] and therefore, \[
\sigma_{Y_{1},Y_{2}} = \rho_{Y_{1},Y_{2}}\sigma_{Y_{1}}\sigma_{Y_{2}},
\] Remember that the covariance matrix is given by \[
\Sigma=
\begin{pmatrix}
\sigma_{Y_{1}}^2 & \sigma_{Y_{1},Y_{2}}\\
\sigma_{Y_{1},Y_{2}} & \sigma_{Y_{2}}^2
\end{pmatrix}.
\] We can now generate the data, which we do as illustrated below. I
will consider \(n=100\) and will also fix the seed (for the random
number generator), so that the results are reproducible (i.e., we will
all obtain the same values). Vetor média e vetor de covariância! E
correlação de 0.6!

Simulamos 100 valores desta distribuição!

\begin{Shaded}
\begin{Highlighting}[]
\FunctionTok{set.seed}\NormalTok{(}\DecValTok{1}\NormalTok{)}
\NormalTok{n }\OtherTok{\textless{}{-}} \DecValTok{100} 
\NormalTok{mu1 }\OtherTok{\textless{}{-}}\NormalTok{ mu2 }\OtherTok{\textless{}{-}} \DecValTok{120}
\NormalTok{sigma1 }\OtherTok{\textless{}{-}}\NormalTok{ sigma2 }\OtherTok{\textless{}{-}} \DecValTok{20}
\NormalTok{rho }\OtherTok{\textless{}{-}} \FloatTok{0.6}

\CommentTok{\#covariance matrix}
\NormalTok{Sigma }\OtherTok{\textless{}{-}} \FunctionTok{matrix}\NormalTok{(}\FunctionTok{c}\NormalTok{(sigma1}\SpecialCharTok{\^{}}\DecValTok{2}\NormalTok{, rho}\SpecialCharTok{*}\NormalTok{sigma1}\SpecialCharTok{*}\NormalTok{sigma2, rho}\SpecialCharTok{*}\NormalTok{sigma1}\SpecialCharTok{*}\NormalTok{sigma2, sigma2}\SpecialCharTok{\^{}}\DecValTok{2}\NormalTok{), }\DecValTok{2}\NormalTok{, }\DecValTok{2}\NormalTok{, }\AttributeTok{byrow =}\NormalTok{ T)}

\CommentTok{\#generate Y=(Y1,Y2)}
\NormalTok{Y }\OtherTok{\textless{}{-}} \FunctionTok{mvrnorm}\NormalTok{(n, }\AttributeTok{mu =} \FunctionTok{c}\NormalTok{(mu1, mu2), }\AttributeTok{Sigma =}\NormalTok{ Sigma)}

\CommentTok{\#looking at the first 10 rows of the dataset}
\NormalTok{Y[}\DecValTok{1}\SpecialCharTok{:}\DecValTok{10}\NormalTok{,]}
\end{Highlighting}
\end{Shaded}

\begin{verbatim}
##            [,1]      [,2]
##  [1,] 114.34238 103.24493
##  [2,] 122.90842 123.66181
##  [3,] 113.19935  96.90429
##  [4,] 147.12380 149.95070
##  [5,] 131.74920 120.03963
##  [6,]  89.51592 121.13011
##  [7,] 122.30897 135.12982
##  [8,] 125.06671 141.34840
##  [9,] 126.86363 133.73615
## [10,]  99.49121 129.58289
\end{verbatim}

\begin{Shaded}
\begin{Highlighting}[]
\CommentTok{\#storing and rounding the simulated values in two variables}
\NormalTok{Y1 }\OtherTok{\textless{}{-}} \FunctionTok{round}\NormalTok{(Y[,}\DecValTok{1}\NormalTok{]); Y2 }\OtherTok{\textless{}{-}} \FunctionTok{round}\NormalTok{(Y[,}\DecValTok{2}\NormalTok{])}
\FunctionTok{mean}\NormalTok{(Y1); }\FunctionTok{mean}\NormalTok{(Y2); }\FunctionTok{sd}\NormalTok{(Y1); }\FunctionTok{sd}\NormalTok{(Y2)}
\end{Highlighting}
\end{Shaded}

\begin{verbatim}
## [1] 122.27
\end{verbatim}

\begin{verbatim}
## [1] 121.61
\end{verbatim}

\begin{verbatim}
## [1] 18.1986
\end{verbatim}

\begin{verbatim}
## [1] 18.17863
\end{verbatim}

Let us start imposing missingness on \(Y_2\) under a MCAR mechanism. One
way to do that is to simply select, say \(30\) individuals out of the
\(100\) in our sample.This mechanism is clearly MCAR.

\begin{Shaded}
\begin{Highlighting}[]
\FunctionTok{set.seed}\NormalTok{(}\DecValTok{1}\NormalTok{)}
\NormalTok{ind }\OtherTok{\textless{}{-}} \FunctionTok{sample}\NormalTok{(}\DecValTok{1}\SpecialCharTok{:}\NormalTok{n, }\AttributeTok{size =} \DecValTok{30}\NormalTok{, }\AttributeTok{replace =}\NormalTok{ F)}
\NormalTok{Y2\_MCAR\_obs }\OtherTok{\textless{}{-}}\NormalTok{ Y2[ind]}
\NormalTok{ind; Y2\_MCAR\_obs}
\end{Highlighting}
\end{Shaded}

\begin{verbatim}
##  [1] 68 39  1 34 87 43 14 82 59 51 85 21 54 74  7 73 79 37 83 97 44 84 33 35 70
## [26] 96 42 38 20 28
\end{verbatim}

\begin{verbatim}
##  [1] 133 134 103 105 152 118  75 126 118 131 135 132  91 103 135 135 131 110 143
## [20] 110 126  80 132  98 161 121 126 114 129  93
\end{verbatim}

\begin{Shaded}
\begin{Highlighting}[]
\FunctionTok{mean}\NormalTok{(Y2\_MCAR\_obs); }\FunctionTok{sd}\NormalTok{(Y2\_MCAR\_obs)}
\end{Highlighting}
\end{Shaded}

\begin{verbatim}
## [1] 120
\end{verbatim}

\begin{verbatim}
## [1] 20.10318
\end{verbatim}

Let us now look at the density plots of the observed, complete, and
missing data (remember that we can only do this because we are working
with simulated data and so that we have access to the complete data and
to the missing data as well).

\begin{Shaded}
\begin{Highlighting}[]
\NormalTok{Y2\_MCAR\_mis }\OtherTok{\textless{}{-}}\NormalTok{ Y2[}\SpecialCharTok{{-}}\NormalTok{ind] }\CommentTok{\#storing the "missing" Y2 values}
\FunctionTok{plot}\NormalTok{(}\FunctionTok{density}\NormalTok{(Y2), }\AttributeTok{lwd =} \DecValTok{2}\NormalTok{, }\AttributeTok{col =} \StringTok{"blue"}\NormalTok{, }\AttributeTok{xlab =} \StringTok{"SBP"}\NormalTok{, }\AttributeTok{main =} \StringTok{"MCAR"}\NormalTok{)}
\FunctionTok{lines}\NormalTok{(}\FunctionTok{density}\NormalTok{(Y2\_MCAR\_obs), }\AttributeTok{lwd =} \DecValTok{2}\NormalTok{, }\AttributeTok{col =} \StringTok{"red"}\NormalTok{)}
\FunctionTok{lines}\NormalTok{(}\FunctionTok{density}\NormalTok{(Y2\_MCAR\_mis), }\AttributeTok{lwd =} \DecValTok{2}\NormalTok{, }\AttributeTok{col =} \StringTok{"darkgreen"}\NormalTok{)}
\FunctionTok{legend}\NormalTok{(}\DecValTok{145}\NormalTok{, }\FloatTok{0.02}\NormalTok{, }\AttributeTok{legend =} \FunctionTok{c}\NormalTok{(}\StringTok{"Complete data"}\NormalTok{, }\StringTok{"Observed data"}\NormalTok{, }\StringTok{"Missing data"}\NormalTok{), }
       \AttributeTok{col =} \FunctionTok{c}\NormalTok{(}\StringTok{"blue"}\NormalTok{, }\StringTok{"red"}\NormalTok{, }\StringTok{"darkgreen"}\NormalTok{), }\AttributeTok{lty =} \FunctionTok{c}\NormalTok{(}\DecValTok{1}\NormalTok{,}\DecValTok{1}\NormalTok{,}\DecValTok{1}\NormalTok{), }\AttributeTok{lwd =} \FunctionTok{c}\NormalTok{(}\DecValTok{2}\NormalTok{,}\DecValTok{2}\NormalTok{,}\DecValTok{2}\NormalTok{), }\AttributeTok{bty =}\StringTok{"n"}\NormalTok{)}
\end{Highlighting}
\end{Shaded}

\begin{center}\includegraphics{Aula-2---parte-1.2_modifications_files/figure-latex/unnamed-chunk-4-1} \end{center}

We can see that the three distributions are very similar, which makes
sense, as the data were generated under the MCAR assumption. If you try
to increase the sample size, e.g., from \(n=100\) to \(n=1000\) and
select \(300\) rather than \(30\) individuals, you will notice that the
distributions become much more similar. We could do a similar
visualisation by using a boxplot.

\begin{Shaded}
\begin{Highlighting}[]
\NormalTok{n\_obs }\OtherTok{\textless{}{-}} \FunctionTok{length}\NormalTok{(Y2\_MCAR\_obs)}
\NormalTok{n\_mis }\OtherTok{\textless{}{-}} \FunctionTok{length}\NormalTok{(Y2\_MCAR\_mis)}
\NormalTok{index }\OtherTok{\textless{}{-}} \FunctionTok{rep}\NormalTok{(}\StringTok{"Y2\_comp"}\NormalTok{, n }\SpecialCharTok{+}\NormalTok{ n\_obs }\SpecialCharTok{+}\NormalTok{ n\_mis)}
\NormalTok{index[(n}\SpecialCharTok{+}\DecValTok{1}\NormalTok{)}\SpecialCharTok{:}\NormalTok{(n}\SpecialCharTok{+}\NormalTok{n\_obs)] }\OtherTok{\textless{}{-}} \StringTok{"Y2\_obs"}
\NormalTok{index[(n}\SpecialCharTok{+}\NormalTok{n\_obs}\SpecialCharTok{+}\DecValTok{1}\NormalTok{)}\SpecialCharTok{:}\NormalTok{(n}\SpecialCharTok{+}\NormalTok{n\_obs}\SpecialCharTok{+}\NormalTok{n\_mis)] }\OtherTok{\textless{}{-}} \StringTok{"Y2\_mis"}
\NormalTok{index1 }\OtherTok{\textless{}{-}} \FunctionTok{factor}\NormalTok{(index, }\AttributeTok{levels =} \FunctionTok{c}\NormalTok{(}\StringTok{"Y2\_comp"}\NormalTok{,}\StringTok{"Y2\_obs"}\NormalTok{,}\StringTok{"Y2\_mis"}\NormalTok{))}
\NormalTok{Y2boxmcar }\OtherTok{\textless{}{-}} \FunctionTok{c}\NormalTok{(Y2, Y2\_MCAR\_obs,Y2\_MCAR\_mis)}
\FunctionTok{boxplot}\NormalTok{(Y2boxmcar }\SpecialCharTok{\textasciitilde{}}\NormalTok{ index1, }\AttributeTok{boxwex =} \FloatTok{0.25}\NormalTok{, }\AttributeTok{col =} \StringTok{"grey"}\NormalTok{, }\AttributeTok{cex.lab =} \FloatTok{0.8}\NormalTok{, }
        \AttributeTok{cex.axis =} \FloatTok{1.2}\NormalTok{, }\AttributeTok{ylab =} \FunctionTok{expression}\NormalTok{(Y[}\DecValTok{2}\NormalTok{]), }\AttributeTok{xlab =}\StringTok{"MCAR"}\NormalTok{)}
\end{Highlighting}
\end{Shaded}

\begin{center}\includegraphics{Aula-2---parte-1.2_modifications_files/figure-latex/unnamed-chunk-5-1} \end{center}

We can use also the function \texttt{pbox} in \texttt{VIM} as
illustrated below, where \(Y_1\) is stratified on the basis of the
missingness in \(Y_2\). Again, there is no evidence that the two induced
groups are different.

\begin{Shaded}
\begin{Highlighting}[]
\NormalTok{Y2\_MCAR\_NA }\OtherTok{\textless{}{-}} \FunctionTok{c}\NormalTok{(Y2\_MCAR\_obs, }\FunctionTok{rep}\NormalTok{(}\ConstantTok{NA}\NormalTok{, n\_mis))}
\NormalTok{df\_NA }\OtherTok{\textless{}{-}} \FunctionTok{data.frame}\NormalTok{(}\StringTok{"Y1"} \OtherTok{=}\NormalTok{ Y1, }\StringTok{"Y2"} \OtherTok{=}\NormalTok{ Y2\_MCAR\_NA)}
\FunctionTok{require}\NormalTok{(VIM)}
\FunctionTok{pbox}\NormalTok{(df\_NA, }\AttributeTok{pos =} \DecValTok{1}\NormalTok{)}
\end{Highlighting}
\end{Shaded}

\begin{center}\includegraphics{Aula-2---parte-1.2_modifications_files/figure-latex/unnamed-chunk-6-1} \end{center}

We will now, only as an example, perform a t-test to check the
plausibility of the MCAR assumption. The null hypothesis is that the
means in the two \(Y_1\) groups formed by stratifying on the basis of
missingness in \(Y_2\) are equal. We therefore should expect not to
reject the null hypothesis in this case. But again, I reiterate, we need
to acknowledge that we are working with a small sample size, and so
results may not turned out as expected. So, for the above generated MCAR
data we will now conduct the unpaired t-test. For more information, type
\texttt{help(t.test)}.

\begin{Shaded}
\begin{Highlighting}[]
\NormalTok{g1 }\OtherTok{\textless{}{-}}\NormalTok{ Y1[}\SpecialCharTok{{-}}\NormalTok{ind]; g2 }\OtherTok{\textless{}{-}}\NormalTok{ Y1[ind]}
\FunctionTok{mean}\NormalTok{(g1); }\FunctionTok{mean}\NormalTok{(g2)}
\end{Highlighting}
\end{Shaded}

\begin{verbatim}
## [1] 122.2286
\end{verbatim}

\begin{verbatim}
## [1] 122.3667
\end{verbatim}

\begin{Shaded}
\begin{Highlighting}[]
\FunctionTok{t.test}\NormalTok{(g1, g2, }\AttributeTok{paired =} \ConstantTok{FALSE}\NormalTok{, }\AttributeTok{var.equal =} \ConstantTok{FALSE}\NormalTok{)}
\end{Highlighting}
\end{Shaded}

\begin{verbatim}
## 
##  Welch Two Sample t-test
## 
## data:  g1 and g2
## t = -0.032984, df = 49.542, p-value = 0.9738
## alternative hypothesis: true difference in means is not equal to 0
## 95 percent confidence interval:
##  -8.549290  8.273099
## sample estimates:
## mean of x mean of y 
##  122.2286  122.3667
\end{verbatim}

We obtain a p-value much larger than \(0.05\) and thus we do not reject
the null hypothesis.

We will now impose a MAR mechanism, and we will do that in the following
way: those who have measurements in February are those whose January's
measurement exceed 140 (i.e., \(Y_1\)\textgreater140), a threshold used
for diagnosing high blood pressure or hypertension.

\begin{Shaded}
\begin{Highlighting}[]
\NormalTok{Y2\_MAR\_obs }\OtherTok{\textless{}{-}}\NormalTok{ Y2[Y1 }\SpecialCharTok{\textgreater{}} \DecValTok{140}\NormalTok{] }
\FunctionTok{mean}\NormalTok{(Y2\_MAR\_obs); }\FunctionTok{sd}\NormalTok{(Y2\_MAR\_obs)}
\end{Highlighting}
\end{Shaded}

\begin{verbatim}
## [1] 138
\end{verbatim}

\begin{verbatim}
## [1] 13.79372
\end{verbatim}

We can now also plot the densities of the complete, observed, and
missing data. And as can be appreciated below the complete and observed
data distributions are now quite different under data generated
according to the MAR mechanism.

\begin{Shaded}
\begin{Highlighting}[]
\NormalTok{Y2\_MAR\_mis }\OtherTok{\textless{}{-}}\NormalTok{ Y2[Y1 }\SpecialCharTok{\textless{}} \DecValTok{140}\NormalTok{] }
\FunctionTok{plot}\NormalTok{(}\FunctionTok{density}\NormalTok{(Y2), }\AttributeTok{lwd =} \DecValTok{2}\NormalTok{, }\AttributeTok{col =} \StringTok{"blue"}\NormalTok{, }\AttributeTok{xlab =} \StringTok{"SBP"}\NormalTok{, }\AttributeTok{main =} \StringTok{"MAR"}\NormalTok{, }\AttributeTok{ylim =} \FunctionTok{c}\NormalTok{(}\DecValTok{0}\NormalTok{,}\FloatTok{0.035}\NormalTok{))}
\FunctionTok{lines}\NormalTok{(}\FunctionTok{density}\NormalTok{(Y2\_MAR\_obs), }\AttributeTok{lwd =} \DecValTok{2}\NormalTok{, }\AttributeTok{col =} \StringTok{"red"}\NormalTok{)}
\FunctionTok{lines}\NormalTok{(}\FunctionTok{density}\NormalTok{(Y2\_MAR\_mis), }\AttributeTok{lwd =} \DecValTok{2}\NormalTok{, }\AttributeTok{col =} \StringTok{"darkgreen"}\NormalTok{)}
\FunctionTok{legend}\NormalTok{(}\DecValTok{145}\NormalTok{, }\FloatTok{0.035}\NormalTok{, }\AttributeTok{legend =} \FunctionTok{c}\NormalTok{(}\StringTok{"Complete data"}\NormalTok{, }\StringTok{"Observed data"}\NormalTok{, }\StringTok{"Missing data"}\NormalTok{), }
       \AttributeTok{col =} \FunctionTok{c}\NormalTok{(}\StringTok{"blue"}\NormalTok{, }\StringTok{"red"}\NormalTok{, }\StringTok{"darkgreen"}\NormalTok{), }\AttributeTok{lty =} \FunctionTok{c}\NormalTok{(}\DecValTok{1}\NormalTok{,}\DecValTok{1}\NormalTok{,}\DecValTok{1}\NormalTok{), }\AttributeTok{lwd =} \FunctionTok{c}\NormalTok{(}\DecValTok{2}\NormalTok{,}\DecValTok{2}\NormalTok{,}\DecValTok{2}\NormalTok{), }\AttributeTok{bty =}\StringTok{"n"}\NormalTok{)}
\end{Highlighting}
\end{Shaded}

\includegraphics{Aula-2---parte-1.2_modifications_files/figure-latex/unnamed-chunk-9-1.pdf}

With only an illustrative purpose, we conduct now again a \(t\)-test,
and given that the missing values were now generated under the MAR
assumption, we expect to reject the null hypothesis of equality of
means.

\begin{Shaded}
\begin{Highlighting}[]
\NormalTok{ind\_MAR }\OtherTok{\textless{}{-}} \FunctionTok{which}\NormalTok{(Y1 }\SpecialCharTok{\textgreater{}} \DecValTok{140}\NormalTok{)}
\NormalTok{g1\_MAR }\OtherTok{\textless{}{-}}\NormalTok{ Y1[ind\_MAR]; g2\_MAR }\OtherTok{\textless{}{-}}\NormalTok{ Y1[}\SpecialCharTok{{-}}\NormalTok{ind\_MAR]}
\FunctionTok{mean}\NormalTok{(g1\_MAR); }\FunctionTok{mean}\NormalTok{(g2\_MAR)}
\end{Highlighting}
\end{Shaded}

\begin{verbatim}
## [1] 151.125
\end{verbatim}

\begin{verbatim}
## [1] 116.7738
\end{verbatim}

\begin{Shaded}
\begin{Highlighting}[]
\FunctionTok{t.test}\NormalTok{(g1\_MAR, g2\_MAR, }\AttributeTok{paired =} \ConstantTok{FALSE}\NormalTok{, }\AttributeTok{var.equal =} \ConstantTok{FALSE}\NormalTok{)}
\end{Highlighting}
\end{Shaded}

\begin{verbatim}
## 
##  Welch Two Sample t-test
## 
## data:  g1_MAR and g2_MAR
## t = 14.158, df = 37.691, p-value < 2.2e-16
## alternative hypothesis: true difference in means is not equal to 0
## 95 percent confidence interval:
##  29.43803 39.26435
## sample estimates:
## mean of x mean of y 
##  151.1250  116.7738
\end{verbatim}

The p-value is much smaller than \(0.05\) and therefore we reject the
null hypothesis of equality of means.

Finally, to induce a MNAR mechanism, the following was implemented: all
individuals with measurements in February are those whose February
measurement itself exceeded 140. This could happen, for instance, if all
individuals had their measurements in February but the staff person only
recorded it in case it was in the hypertensive range. MNAR could be
induced in other ways, e.g., February measurement only recorded if it is
substantial different from the January one. This is an example where the
mising values depend both on the missing and observed values (because
they depend on the difference between \(Y_1\) and \(Y_2\)).

\begin{Shaded}
\begin{Highlighting}[]
\NormalTok{Y2\_MNAR\_obs }\OtherTok{\textless{}{-}}\NormalTok{ Y2[Y2 }\SpecialCharTok{\textgreater{}} \DecValTok{140}\NormalTok{] }
\FunctionTok{mean}\NormalTok{(Y2\_MNAR\_obs); }\FunctionTok{sd}\NormalTok{(Y2\_MNAR\_obs)}
\end{Highlighting}
\end{Shaded}

\begin{verbatim}
## [1] 147.7333
\end{verbatim}

\begin{verbatim}
## [1] 7.41106
\end{verbatim}

\begin{Shaded}
\begin{Highlighting}[]
\NormalTok{Y2\_MNAR\_mis }\OtherTok{\textless{}{-}}\NormalTok{ Y2[Y2 }\SpecialCharTok{\textless{}} \DecValTok{140}\NormalTok{] }
\FunctionTok{plot}\NormalTok{(}\FunctionTok{density}\NormalTok{(Y2), }\AttributeTok{lwd =} \DecValTok{2}\NormalTok{, }\AttributeTok{col =} \StringTok{"blue"}\NormalTok{, }\AttributeTok{xlab =} \StringTok{"SBP"}\NormalTok{, }\AttributeTok{main =} \StringTok{"MNAR"}\NormalTok{, }\AttributeTok{ylim =} \FunctionTok{c}\NormalTok{(}\DecValTok{0}\NormalTok{,}\FloatTok{0.08}\NormalTok{))}
\FunctionTok{lines}\NormalTok{(}\FunctionTok{density}\NormalTok{(Y2\_MNAR\_obs), }\AttributeTok{lwd =} \DecValTok{2}\NormalTok{, }\AttributeTok{col =} \StringTok{"red"}\NormalTok{)}
\FunctionTok{lines}\NormalTok{(}\FunctionTok{density}\NormalTok{(Y2\_MNAR\_mis), }\AttributeTok{lwd =} \DecValTok{2}\NormalTok{, }\AttributeTok{col =} \StringTok{"darkgreen"}\NormalTok{)}
\FunctionTok{legend}\NormalTok{(}\DecValTok{145}\NormalTok{, }\FloatTok{0.08}\NormalTok{, }\AttributeTok{legend =} \FunctionTok{c}\NormalTok{(}\StringTok{"Complete data"}\NormalTok{, }\StringTok{"Observed data"}\NormalTok{, }\StringTok{"Missing data"}\NormalTok{), }
       \AttributeTok{col =} \FunctionTok{c}\NormalTok{(}\StringTok{"blue"}\NormalTok{, }\StringTok{"red"}\NormalTok{, }\StringTok{"darkgreen"}\NormalTok{), }\AttributeTok{lty =} \FunctionTok{c}\NormalTok{(}\DecValTok{1}\NormalTok{,}\DecValTok{1}\NormalTok{,}\DecValTok{1}\NormalTok{), }\AttributeTok{lwd =} \FunctionTok{c}\NormalTok{(}\DecValTok{2}\NormalTok{,}\DecValTok{2}\NormalTok{,}\DecValTok{2}\NormalTok{), }\AttributeTok{bty =}\StringTok{"n"}\NormalTok{)}
\end{Highlighting}
\end{Shaded}

\includegraphics{Aula-2---parte-1.2_modifications_files/figure-latex/unnamed-chunk-11-1.pdf}

In this example we can notice that as we move from MCAR to MAR to MNAR,
the observed \(Y_2\) values become an increasingly select and unusual
group relative to the complete data. Although this phenomenon is not a
universal feature of MCAR, MAR, and MNAR, it does happen in many
realistic examples.

\hypertarget{numerical-example-from-van-buuren-2nd-edition-p.-37}{%
\section{Numerical example from van Buuren (2nd edition,
p.~37)}\label{numerical-example-from-van-buuren-2nd-edition-p.-37}}

The aim of this example is also to simulate data from MCAR, MAR and MNAR
mechanisms and it is, in essence, quite similar to the previous one. Let
the data \(Y=(Y_1,Y_2)\) be simulated from a standard bivariate normal
distribution with correlation \(\rho_{Y_1,Y_2}=0.5\). Missing data are
created in \(Y_2\) using the missing data model \[
\Pr(R=0\mid Y_1, Y_2,\psi)=\psi_0+\frac{e^{Y_1}}{1+e^{Y_1}}\psi_1+\frac{e^{Y_2}}{1+e^{Y_2}}\psi_2,
\] with different parameter settings for
\(\psi=(\psi_0,\psi_1,\psi_2)\). For MCAR we set
\(\psi_{\text{MCAR}}=(0.5,0,0)\), for MAR we set
\(\psi_{\text{MAR}}=(0,1,0)\), and for MNAR we set
\(\psi_{\text{MNAR}}=(0,0,1)\). Thus, we obtain the following models:

\begin{itemize}
\item
  MCAR: \(\Pr(R=0\mid Y_1, Y_2)=0.5\).
\item
  MAR: \(\Pr(R=0\mid Y_1, Y_2)=\frac{e^{Y_1}}{1+e^{Y_1}}\) or,
  equivalently, \(\text{logit}\{\Pr(R=0)\}=Y_1\).
\item
  MNAR: \(\text{logit}\{\Pr(R=0\mid Y_1, Y_2)\}=Y_2\).
\end{itemize}

Here, \(\text{logit}(p)=\log\{p/(1-p)\}\), for \(0<p<1\), is the logit
function. Note also that since only one variable has missing values, one
missingness indicator suffices.

We start by generating the data and following the author we will
generate \(300\) observations.

\begin{Shaded}
\begin{Highlighting}[]
\FunctionTok{require}\NormalTok{(MASS)}
\FunctionTok{set.seed}\NormalTok{(}\DecValTok{1}\NormalTok{)}
\NormalTok{n }\OtherTok{\textless{}{-}} \DecValTok{300}
\NormalTok{Sigma }\OtherTok{\textless{}{-}} \FunctionTok{matrix}\NormalTok{(}\FunctionTok{c}\NormalTok{(}\DecValTok{1}\NormalTok{,}\FloatTok{0.5}\NormalTok{,}\FloatTok{0.5}\NormalTok{,}\DecValTok{1}\NormalTok{), }\AttributeTok{nrow =} \DecValTok{2}\NormalTok{, }\AttributeTok{byrow =}\NormalTok{ T)}
\NormalTok{Y }\OtherTok{\textless{}{-}} \FunctionTok{mvrnorm}\NormalTok{(}\AttributeTok{n =}\NormalTok{ n, }\AttributeTok{mu =} \FunctionTok{c}\NormalTok{(}\DecValTok{0}\NormalTok{,}\DecValTok{0}\NormalTok{), }\AttributeTok{Sigma =}\NormalTok{ Sigma)}
\NormalTok{Y1 }\OtherTok{\textless{}{-}}\NormalTok{ Y[,}\DecValTok{1}\NormalTok{]; Y2 }\OtherTok{\textless{}{-}}\NormalTok{ Y[,}\DecValTok{2}\NormalTok{]}
\end{Highlighting}
\end{Shaded}

We will generate the missing data indicator \(R\) (which can only take
the value \(0\) or \(1\)) from a Bernoulli distribution. In \texttt{R}
we can do that through the function \texttt{rbinom} (that generates
random variables from a Binomial distribution) by letting
\texttt{size=1}. We need to pass as input the probability of success
(roughly speaking the probability of obtaining a one), which in our
case, would correspond to \(\Pr(R=1\mid Y_1, Y_2)\). For more
information type \texttt{help(rbinom)}. Note that we are given the
`complementary' probability \(\Pr(R=0\mid Y_1, Y_2)\). Since
\(\Pr(R=1\mid Y_1, Y_2) + \Pr(R_2=0\mid Y_1, Y_2) =1\), we then have
\(\Pr(R=1\mid Y_1, Y_2)=\frac{1}{1+e^{Y_1}}\). We will now impose
missingness on \(Y_2\) according to the three mechanisms stated above.

\begin{Shaded}
\begin{Highlighting}[]
\FunctionTok{set.seed}\NormalTok{(}\DecValTok{1}\NormalTok{)}
\NormalTok{r\_mcar }\OtherTok{\textless{}{-}} \FunctionTok{rbinom}\NormalTok{(}\AttributeTok{n =}\NormalTok{ n, }\AttributeTok{size =} \DecValTok{1}\NormalTok{, }\AttributeTok{prob =} \FloatTok{0.5}\NormalTok{)}
\FunctionTok{set.seed}\NormalTok{(}\DecValTok{1}\NormalTok{)}
\NormalTok{r\_mar }\OtherTok{\textless{}{-}} \FunctionTok{rbinom}\NormalTok{(}\AttributeTok{n =}\NormalTok{ n, }\AttributeTok{size =} \DecValTok{1}\NormalTok{, }\AttributeTok{prob =} \DecValTok{1}\SpecialCharTok{/}\NormalTok{(}\DecValTok{1}\SpecialCharTok{+}\FunctionTok{exp}\NormalTok{(Y1)))}
\FunctionTok{set.seed}\NormalTok{(}\DecValTok{1}\NormalTok{)}
\NormalTok{r\_mnar }\OtherTok{\textless{}{-}} \FunctionTok{rbinom}\NormalTok{(}\AttributeTok{n =}\NormalTok{ n, }\AttributeTok{size =} \DecValTok{1}\NormalTok{, }\AttributeTok{prob =} \DecValTok{1}\SpecialCharTok{/}\NormalTok{(}\DecValTok{1}\SpecialCharTok{+}\FunctionTok{exp}\NormalTok{(Y2)))}
\end{Highlighting}
\end{Shaded}

We will now, as in the blood pressure example, plot the density of the
complete, observed, and missing data for each mechanism.

\begin{Shaded}
\begin{Highlighting}[]
\CommentTok{\#MCAR}
\NormalTok{ind\_mcar\_obs }\OtherTok{\textless{}{-}} \FunctionTok{which}\NormalTok{(r\_mcar }\SpecialCharTok{==} \DecValTok{1}\NormalTok{)}
\NormalTok{Y2\_MCAR\_obs }\OtherTok{\textless{}{-}}\NormalTok{ Y2[ind\_mcar\_obs]}
\NormalTok{Y2\_MCAR\_mis }\OtherTok{\textless{}{-}}\NormalTok{ Y2[}\SpecialCharTok{{-}}\NormalTok{ind\_mcar\_obs]}
\FunctionTok{plot}\NormalTok{(}\FunctionTok{density}\NormalTok{(Y2), }\AttributeTok{lwd =} \DecValTok{2}\NormalTok{, }\AttributeTok{col =} \StringTok{"blue"}\NormalTok{, }\AttributeTok{xlab =} \FunctionTok{expression}\NormalTok{(Y[}\DecValTok{2}\NormalTok{]), }\AttributeTok{main =} \StringTok{"MCAR"}\NormalTok{, }\AttributeTok{ylim =} \FunctionTok{c}\NormalTok{(}\DecValTok{0}\NormalTok{, }\FloatTok{0.6}\NormalTok{))}
\FunctionTok{lines}\NormalTok{(}\FunctionTok{density}\NormalTok{(Y2\_MCAR\_obs), }\AttributeTok{lwd =} \DecValTok{2}\NormalTok{, }\AttributeTok{col =} \StringTok{"red"}\NormalTok{)}
\FunctionTok{lines}\NormalTok{(}\FunctionTok{density}\NormalTok{(Y2\_MCAR\_mis), }\AttributeTok{lwd =} \DecValTok{2}\NormalTok{, }\AttributeTok{col =} \StringTok{"darkgreen"}\NormalTok{)}
\FunctionTok{legend}\NormalTok{(}\FloatTok{1.2}\NormalTok{, }\FloatTok{0.4}\NormalTok{, }\AttributeTok{legend =} \FunctionTok{c}\NormalTok{(}\StringTok{"Complete data"}\NormalTok{, }\StringTok{"Observed data"}\NormalTok{, }\StringTok{"Missing data"}\NormalTok{), }
       \AttributeTok{col =} \FunctionTok{c}\NormalTok{(}\StringTok{"blue"}\NormalTok{, }\StringTok{"red"}\NormalTok{, }\StringTok{"darkgreen"}\NormalTok{), }\AttributeTok{lty =} \FunctionTok{c}\NormalTok{(}\DecValTok{1}\NormalTok{,}\DecValTok{1}\NormalTok{,}\DecValTok{1}\NormalTok{), }\AttributeTok{lwd =} \FunctionTok{c}\NormalTok{(}\DecValTok{2}\NormalTok{,}\DecValTok{2}\NormalTok{,}\DecValTok{2}\NormalTok{), }\AttributeTok{bty =}\StringTok{"n"}\NormalTok{)}
\end{Highlighting}
\end{Shaded}

\includegraphics{Aula-2---parte-1.2_modifications_files/figure-latex/unnamed-chunk-14-1.pdf}

\begin{Shaded}
\begin{Highlighting}[]
\CommentTok{\#MAR}
\NormalTok{ind\_mar\_obs }\OtherTok{\textless{}{-}} \FunctionTok{which}\NormalTok{(r\_mar }\SpecialCharTok{==} \DecValTok{1}\NormalTok{)}
\NormalTok{Y2\_MAR\_obs }\OtherTok{\textless{}{-}}\NormalTok{ Y2[ind\_mar\_obs]}
\NormalTok{Y2\_MAR\_mis }\OtherTok{\textless{}{-}}\NormalTok{ Y2[}\SpecialCharTok{{-}}\NormalTok{ind\_mar\_obs]}
\FunctionTok{plot}\NormalTok{(}\FunctionTok{density}\NormalTok{(Y2), }\AttributeTok{lwd =} \DecValTok{2}\NormalTok{, }\AttributeTok{col =} \StringTok{"blue"}\NormalTok{, }\AttributeTok{xlab =} \FunctionTok{expression}\NormalTok{(Y[}\DecValTok{2}\NormalTok{]), }\AttributeTok{main =} \StringTok{"MAR"}\NormalTok{, }\AttributeTok{ylim =} \FunctionTok{c}\NormalTok{(}\DecValTok{0}\NormalTok{, }\FloatTok{0.6}\NormalTok{))}
\FunctionTok{lines}\NormalTok{(}\FunctionTok{density}\NormalTok{(Y2\_MAR\_obs), }\AttributeTok{lwd =} \DecValTok{2}\NormalTok{, }\AttributeTok{col =} \StringTok{"red"}\NormalTok{)}
\FunctionTok{lines}\NormalTok{(}\FunctionTok{density}\NormalTok{(Y2\_MAR\_mis), }\AttributeTok{lwd =} \DecValTok{2}\NormalTok{, }\AttributeTok{col =} \StringTok{"darkgreen"}\NormalTok{)}
\FunctionTok{legend}\NormalTok{(}\FloatTok{1.2}\NormalTok{, }\FloatTok{0.4}\NormalTok{, }\AttributeTok{legend =} \FunctionTok{c}\NormalTok{(}\StringTok{"Complete data"}\NormalTok{, }\StringTok{"Observed data"}\NormalTok{, }\StringTok{"Missing data"}\NormalTok{), }
       \AttributeTok{col =} \FunctionTok{c}\NormalTok{(}\StringTok{"blue"}\NormalTok{, }\StringTok{"red"}\NormalTok{, }\StringTok{"darkgreen"}\NormalTok{), }\AttributeTok{lty =} \FunctionTok{c}\NormalTok{(}\DecValTok{1}\NormalTok{,}\DecValTok{1}\NormalTok{,}\DecValTok{1}\NormalTok{), }\AttributeTok{lwd =} \FunctionTok{c}\NormalTok{(}\DecValTok{2}\NormalTok{,}\DecValTok{2}\NormalTok{,}\DecValTok{2}\NormalTok{), }\AttributeTok{bty =}\StringTok{"n"}\NormalTok{)}
\end{Highlighting}
\end{Shaded}

\includegraphics{Aula-2---parte-1.2_modifications_files/figure-latex/unnamed-chunk-14-2.pdf}

\begin{Shaded}
\begin{Highlighting}[]
\CommentTok{\#MNAR}
\NormalTok{ind\_mnar\_obs }\OtherTok{\textless{}{-}} \FunctionTok{which}\NormalTok{(r\_mnar }\SpecialCharTok{==} \DecValTok{1}\NormalTok{)}
\NormalTok{Y2\_MNAR\_obs }\OtherTok{\textless{}{-}}\NormalTok{ Y2[ind\_mnar\_obs]}
\NormalTok{Y2\_MNAR\_mis }\OtherTok{\textless{}{-}}\NormalTok{ Y2[}\SpecialCharTok{{-}}\NormalTok{ind\_mnar\_obs]}
\FunctionTok{plot}\NormalTok{(}\FunctionTok{density}\NormalTok{(Y2), }\AttributeTok{lwd =} \DecValTok{2}\NormalTok{, }\AttributeTok{col =} \StringTok{"blue"}\NormalTok{, }\AttributeTok{xlab =} \FunctionTok{expression}\NormalTok{(Y[}\DecValTok{2}\NormalTok{]), }\AttributeTok{main =} \StringTok{"MNAR"}\NormalTok{, }\AttributeTok{ylim =} \FunctionTok{c}\NormalTok{(}\DecValTok{0}\NormalTok{, }\FloatTok{0.6}\NormalTok{))}
\FunctionTok{lines}\NormalTok{(}\FunctionTok{density}\NormalTok{(Y2\_MNAR\_obs), }\AttributeTok{lwd =} \DecValTok{2}\NormalTok{, }\AttributeTok{col =} \StringTok{"red"}\NormalTok{)}
\FunctionTok{lines}\NormalTok{(}\FunctionTok{density}\NormalTok{(Y2\_MNAR\_mis), }\AttributeTok{lwd =} \DecValTok{2}\NormalTok{, }\AttributeTok{col =} \StringTok{"darkgreen"}\NormalTok{)}
\FunctionTok{legend}\NormalTok{(}\FloatTok{1.2}\NormalTok{, }\FloatTok{0.4}\NormalTok{, }\AttributeTok{legend =} \FunctionTok{c}\NormalTok{(}\StringTok{"Complete data"}\NormalTok{, }\StringTok{"Observed data"}\NormalTok{, }\StringTok{"Missing data"}\NormalTok{), }
       \AttributeTok{col =} \FunctionTok{c}\NormalTok{(}\StringTok{"blue"}\NormalTok{, }\StringTok{"red"}\NormalTok{, }\StringTok{"darkgreen"}\NormalTok{), }\AttributeTok{lty =} \FunctionTok{c}\NormalTok{(}\DecValTok{1}\NormalTok{,}\DecValTok{1}\NormalTok{,}\DecValTok{1}\NormalTok{), }\AttributeTok{lwd =} \FunctionTok{c}\NormalTok{(}\DecValTok{2}\NormalTok{,}\DecValTok{2}\NormalTok{,}\DecValTok{2}\NormalTok{), }\AttributeTok{bty =}\StringTok{"n"}\NormalTok{)}
\end{Highlighting}
\end{Shaded}

\includegraphics{Aula-2---parte-1.2_modifications_files/figure-latex/unnamed-chunk-14-3.pdf}

As for the previous example, although not to the same extent, when
moving from MCAR to MAR and to MNAR, the distributions of the complete,
observed, and missing data become more distinct.

\begin{Shaded}
\begin{Highlighting}[]
\FunctionTok{sessionInfo}\NormalTok{()}
\end{Highlighting}
\end{Shaded}

\begin{verbatim}
## R version 4.3.0 (2023-04-21 ucrt)
## Platform: x86_64-w64-mingw32/x64 (64-bit)
## Running under: Windows 11 x64 (build 22631)
## 
## Matrix products: default
## 
## 
## locale:
## [1] LC_COLLATE=English_Austria.utf8  LC_CTYPE=English_Austria.utf8   
## [3] LC_MONETARY=English_Austria.utf8 LC_NUMERIC=C                    
## [5] LC_TIME=English_Austria.utf8    
## 
## time zone: Europe/Lisbon
## tzcode source: internal
## 
## attached base packages:
## [1] grid      stats     graphics  grDevices utils     datasets  methods  
## [8] base     
## 
## other attached packages:
## [1] VIM_6.2.2        colorspace_2.1-0 MASS_7.3-58.4   
## 
## loaded via a namespace (and not attached):
##  [1] cli_3.6.1           knitr_1.43          rlang_1.1.1        
##  [4] xfun_0.39           highr_0.10          DEoptimR_1.1-3-1   
##  [7] car_3.1-2           data.table_1.14.8   zoo_1.8-12         
## [10] ranger_0.17.0       htmltools_0.5.5     e1071_1.7-13       
## [13] nnet_7.3-18         lmtest_0.9-40       sp_2.0-0           
## [16] rmarkdown_2.22      evaluate_0.21       carData_3.0-5      
## [19] abind_1.4-5         fastmap_1.1.1       yaml_2.3.7         
## [22] compiler_4.3.0      robustbase_0.99-4-1 Rcpp_1.0.10        
## [25] vcd_1.4-13          rstudioapi_0.15.0   lattice_0.21-8     
## [28] digest_0.6.31       laeken_0.5.3        class_7.3-22       
## [31] Matrix_1.5-4        tools_4.3.0         proxy_0.4-27       
## [34] boot_1.3-28.1
\end{verbatim}

\end{document}
